\newword{WYSIWYG}{``What You See Is What You Get''  ou ``O que você vê é o que você obtém''.  Recurso tem por objetivo permitir que um documento, enquanto manipulado na tela, tenha a mesma aparência de sua utilização, usualmente sendo considerada final. Isso facilita para o desenvolvedor que pode trabalhar visualizando a aparência do documento sem precisar salvar em vários momentos e abrir em um \textit{software} separado de visualização}
\newword{Framework}{é uma abstração que une códigos comuns entre vários projetos de \textit{software} provendo uma funcionalidade genérica. \textit{Frameworks} são projetados com a intenção de facilitar o desenvolvimento de \textit{software}, habilitando designers e programadores a gastarem mais tempo determinando as exigências do \textit{software} do que com detalhes de baixo nível do sistema}

\newword{Template}{é um documento sem conteúdo, com apenas a apresentação visual (apenas cabeçalhos por exemplo) e instruções sobre onde e qual tipo de conteúdo deve entrar a cada parcela da apresentação}

\newword{Padrões de projeto}{ou \textit{Design Pattern}, descreve uma solução geral reutilizável para um problema recorrente no desenvolvimento de sistemas de \textit{software} orientados a objetos. Não é um código final, é uma descrição ou modelo de como resolver o problema do qual trata, que pode ser usada em muitas situações diferentes}

\newword{Web}{Sinônimo mais conhecido de \textit{World Wide Web} (WWW). É a interface gráfica da Internet que torna os serviços disponíveis totalmente transparentes para o usuário e ainda possibilita a manipulação multimídia da informação}
